\documentclass[%
 %twocolumn,
 preprint,
 amsmath, amssymb, aps, pra, 10pt
]{revtex4-2}

\usepackage{amsmath}

\begin{document}

\title{Fractional Calculus}% Force line breaks with \

\author{Luke A. Siemens}
\email{luke.siemens@lsiemens.com}
\noaffiliation

\date{\today}

\maketitle

\section{Goal}
find a sub set of functions, and set of properties for which a unique fractional exists with a nice algebra.

\section{motivation, introduction}
discus what I find unsatisfying about fractional calculus, mainly  the multiple incompatible definitions and the semigroup algebra of the operatiors.

\section{main body}
%spesify the target properties of this fractional calculus, linear operator, group algebra? invertable, analytic in the operator parameter $\alpha$.

Let the proposed fractional calculus operator $J^\alpha$ have the folowing properties, for $\alpha \in \mathbb{Z}$ it can reporoduce repeated integration and differntiation, it is a linear operator, that $J^\alpha$ forms an abelian group with the field $\mathbb{C}$ and group operator $J^\alpha J^\beta$ when acting on some sutable subset of differentiable functions. It is analytic in the parameter $\alpha$.

\noindent\rule{\textwidth}{1pt}

%find calculus operator with the proper algebra for the extension when acting on some subset of functions $\mathbb{S} \subset C^{\infty}(\mathbb{R})$, operators $\frac{d^n}{dx^n}$, $\int_{-\infty}^x f(t)dt$.

First let us identify a sutable operator to extend into a fractional calculus with nice algebra, and function space for which this operator is well behaved. Let the operator $J^n$ for $n \in \mathbb{Z}$ be,

\begin{equation}
J^n f(x) := \begin{cases} \frac{1}{n!}\int_{-\infty}^x (x - t)^{n - 1}f(t)dt & n \geq 1 \\ f(x) & n = 0 \\ \frac{d^{-n}}{dx^{-n}}f(x) & n \leq -1 \end{cases}
\label{integer_calculus}
\end{equation}

Using this operator one space of functions where the operator is well behaved is any function that is bounded by positive exponentials lets call this set $\mathbb{S}$,

\begin{equation}
\mathbb{S} := \left\lbrace f \in C^\infty(\mathbb{R}) \middle| (\forall n \in \mathbb{Z}^+)(\exists a_n \in \mathbb{M}, b_n \in \mathbb{R}) b_n > 0, \frac{d^n}{dx^n}f(x) \in \mathbb{B}(a_n, b_n) \right\rbrace
\label{exponentialy_bounded}
\end{equation}

The sets $\mathbb{M}$ and $\mathbb{B}$ are defined as follows,

\[\mathbb{M} := \left\lbrace a \in C(\mathbb{R}) \middle| (\forall x, y \in \mathbb{R}) x > y, a(x) > a(y) \geq 0 \right\rbrace\]

\[\mathbb{B}(a, b) := \left\lbrace f \in C^\infty(\mathbb{R}) \middle| (\forall x, x_0 \in \mathbb{R}) x \leq x_0, |f(x)| \leq a(x_0)e^{bx} \right\rbrace\]

notice that the operator $J^n$ acting on functions $f(x) \in \mathbb{S}$ is an abelian group with respect to the parameter $n$ and is a linear operator.

\noindent\rule{\textwidth}{1pt}

%use ramanujan master theorem to get $\frac{d^{-\alpha}}{dx^{-\alpha}}$ from $\frac{d^n}{dx^n}$, analyticaly continu the function for the full operator
Now that we have a sutable linear operator to extend, we need a procedure to actualy extend the operator. We will use Ramanujan's Master Theorem (RMT) for this purpose, it can be stated as,

\[g(u) = \sum_{k=0}^\infty \frac{\phi(k)(-u)^k}{k!}\]

Given appropriate conditions on $\phi(k)$ the summ converges and the folowing result holds,

\[\int_0^{\infty} u^{s-1}g(u)du = \Gamma(s)\phi(-s)\]

When applicable this theorem acts to interpolate the sequence $\phi(k), k \in \mathbb{Z}^+$, finding an analytic function $\phi(-s)$ reproducing the sequence when $s = -k, k \in \mathbb{Z}^+$. Let us define $\phi(k)$ interms of a function $f(x) \in C^{\omega}(\mathbb{C}), f(x) \in \mathbb{S}$ ,

\[\phi(k) = \left. J^{-k}f(x)\right|_{x = x_0} = \left. \frac{d^k}{dx^k}f(x) \right|_{x = x_0} = F(x_0, -k)\]

then in this case $g(u)$ is,

\[g(u) = \sum_{k=0}^\infty \frac{(-u)^k}{k!} \left. \frac{d^k}{dx^k}f(x)\right|_{x=x_0}\]

recognize that $g(u)$ is the taylor expansion of $f(x_0 - u)$ in terms of $u$. Now using $f(x_0 - u)$ in the integral,

\[\int_0^{\int} u^{s-1}f(x_0 - u)du = \Gamma(s)\phi(-s) = \Gamma(s)F(x_0, s)\]

using the substitution $t = x_0 - u$, the integral becomes,


\[\int_{x_0}^{-\infty} -(x_0 - t)^{s-1}f(t)dt = \Gamma(s)F(x_0, s)\]

and finally rearanging,

\[F(x_0, s) = \frac{1}{\Gamma(s)} \int_{-\infty}^{x_0} (x_0 - t)^{s-1} f(t)dt\]

The function $F(x, s)$ as defined using the (RMT) is only defined on the region $\mathfrak{R}(s) \geq 1$, the function can be extended using the observation that

\[F(x, s-1) = \frac{1}{\Gamma(s)} \int_{-\infty}^{x} (x - t)^{s-1} F(t, -1)dt = \frac{1}{\Gamma(s)} \int_{-\infty}^{x} (x - t)^{s-1} \frac{d}{dt}f(t)dt\]

and given the properties of $f(x) \in \mathbb{S}$

\[\frac{d}{dx}F(x, s) = \frac{d}{dx}\frac{1}{\Gamma(s)} \int_{-\infty}^{x} (x - t)^{s-1} f(t)dt = \frac{1}{\Gamma(s - 1)} \int_{-\infty}^{x} (x - t)^{s - 2} f(t)dt = F(x, s - 1)\]

for functions $f(x) \in \mathbb{S}$, let $F(x, s) = J^{s}f(x) = \frac{1}{\Gamma(s)} \int_{-\infty}^{x} (x - t)^{s-1} f(t)dt$
So using RMT to interpolate betwean the derivatives of $f(x)$ yields a definition for a fractional integral operator which is analytic in respect to $s$.

\noindent\rule{\textwidth}{1pt}

%observe the limitiations of the RMT and add one constraint to produce a unique definition

Note that the function $\phi(-s)$ produced from RMT is not unique.

\[\psi(s) \in C^{\omega}(\mathbb{C}), \psi(k) = 0, k \in \mathbb{Z}^+\]
\[\phi'(k) = \psi(k) + \phi(k) = \phi(k), k \in \mathbb{Z}^+\]

So applying RMT to $\phi'(k)$ will yield $\phi(-s)$ and not $\phi(-s) + \psi(-s)$. This demenstrates that while $J^\alpha$ has many of the properties required for fractional calculus it is not unique. Lets denote an arbitrary compatible fractional calculus operator as $I^\alpha$, then

\[R^\alpha f(x) = I^\alpha f(x) - J^\alpha f(x)\]

where $J^\alpha$ is operator found using RMT the RL integral. If we can prove that $R^\alpha = 0$ when some aditional constraint is applied to the definition of fractional calculus, then that constraint would force the operator $I^\alpha$ to be uniquely defined. Apply the generalized Leibniz rule given by \cite{Leibniz} equation (15.11) to the operator $I^\alpha$,

\[I^\alpha f(x)g(x) = \sum_{k=0}^\infty \binom{-\alpha}{k}\left( I^{\alpha + k}f(x) \right)\left( \frac{d^k}{dx^k} g(x)\right) = \sum_{k=0}^\infty \binom{-\alpha}{k}\left( \left(J^{\alpha + k} + R^{\alpha + k}\right)f(x) \right)\left( \frac{d^k}{dx^k} g(x)\right)\]

then subtracting $J^\alpha f(x)g(x)$ from both sides, 

\begin{equation}
R^\alpha f(x)g(x) = \sum_{k=0}^\infty \binom{-\alpha}{k}\left( R^{\alpha + k}f(x) \right)\left( \frac{d^k}{dx^k} g(x)\right)
\label{RemainderOperatorProductRule}
\end{equation}

Now that we are setup, look at the set of ODEs $\frac{d^n}{dx^n}f(x) = f(x)$. All solutions to these equations are summes of one or more exponentials, and inpaticular

\[\exists f(x) \forall n \in \mathbb{Z}, \frac{d^n}{dx^n}f(x) = f(x)\]

where interpreting negetive intagers as repeated integrals of the form $\int_{-\infty}^x f(t)dt$. This statement is true and the only nontrivial $f(x)$ that statisfies it is $f(x) = e^x$ (ignoring the scaling constant). Any ambiguity in the statment can be removed by phrasing it as,

\begin{equation}
\exists f(x) \forall n \in \mathbb{Z}, \frac{d^n}{dx^n}f(x) = f(x), f(0) = 1
\label{ConstaintOnCalculus}
\end{equation}

Let us generalize this statment to fractional calculus,

\begin{equation}
\exists f(x) \forall \alpha \in \mathbb{C}, \frac{d^\alpha}{dx^\alpha}f(x) = f(x), f(0) = 1
\label{ConstaintOnFractionalCalculus}
\end{equation}

From statement \eqref{ConstaintOnFractionalCalculus} we can conclude that if $f(x)$ exists it must be $f(x) = e^x$, since it nesesitates that $\frac{d}{dx}f(x) = f(x), f(0) = 1$. We will now re	quire that statment \eqref{ConstaintOnFractionalCalculus} applies to fractional calculus. So $I^\alpha e^x = J^\alpha e^x + R^\alpha e^x = e^x$, but since $J^\alpha e^x = e^x$ then $R^\alpha e^x = 0$. Now apply this result to \eqref{RemainderOperatorProductRule} with $f(x) = e^x$ and $g(x) = e^{-x}h(x)$ with $h(x)$ in the domain of $I^\alpha$

\[R^\alpha h(x) = \sum_{k=0}^\infty \binom{-\alpha}{k}\left( R^{\alpha + k}e^x \right)\left( \frac{d^k}{dx^k} h(x)e^{-x}\right) = \sum_{k=0}^\infty 0 \binom{-\alpha}{k}\left( \frac{d^k}{dx^k} h(x)e^{-x}\right) = 0\]

So given \eqref{ConstaintOnFractionalCalculus} is true, then $R^\alpha = 0$, and the fractional calculus operator $J^\alpha$ derived from RMT is the only operator satisfying all of our constraints.

\noindent\rule{\textwidth}{1pt}

demenstrate that $\frac{d^{-\alpha}}{dx^{-\alpha}}$ is defined on $\mathbb{S}$ with the required properties

\noindent\rule{\textwidth}{1pt}

\section{Implications and future develepmonts}
briefly show that $\frac{d^{-\alpha}}{dx^{-\alpha}}$ unifies the RL integral the RL derivative the Caputo derivative and the GL derivative

\noindent\rule{\textwidth}{1pt}

mention extensions lik variy  the constrint $\frac{d^{-\alpha}}{dx^{-\alpha}}e^x = e^x$ to produce a broder calculus compatible with the generalized Cauchy fractional derivative and the fourier transform versions of fractional calculus. 

\begin{thebibliography}{1}

\bibitem{Leibniz}
  S.G. Samko, A. A. Kilbas and O. L. Marichev
  \textit{Fractional Integrals and Deriatives},
  Gordon and Breach Science Pubblishers
  1987.

\end{thebibliography}

\end{document}
