% ****** Start of file apssamp.tex ******
%
%   This file is part of the APS files in the REVTeX 4.1 distribution.
%   Version 4.1r of REVTeX, August 2010
%
%   Copyright (c) 2009, 2010 The American Physical Society.
%
%   See the REVTeX 4 README file for restrictions and more information.
%
% TeX'ing this file requires that you have AMS-LaTeX 2.0 installed
% as well as the rest of the prerequisites for REVTeX 4.1
%
% See the REVTeX 4 README file
% It also requires running BibTeX. The commands are as follows:
%
%  1)  latex apssamp.tex
%  2)  bibtex apssamp
%  3)  latex apssamp.tex
%  4)  latex apssamp.tex
%
\documentclass[%
 twocolumn,
% preprint, onecolumn,
 amsmath,amssymb,
 aps,
% floatfix,
]{revtex4-1}

\usepackage{graphicx}% Include figure files
\usepackage{dcolumn}% Align table columns on decimal point
\usepackage{bm}% bold math
\usepackage{amsmath}% better dot placment
\usepackage{systeme}% systemes of equations
%\usepackage{hyperref}% add hypertext capabilities
%\usepackage[mathlines]{lineno}% Enable numbering of text and display math
%\linenumbers\relax % Commence numbering lines

%\usepackage[showframe,%Uncomment any one of the following lines to test 
%%scale=0.7, marginratio={1:1, 2:3}, ignoreall,% default settings
%%text={7in,10in},centering,
%%margin=1.5in,
%%total={6.5in,8.75in}, top=1.2in, left=0.9in, includefoot,
%%height=10in,a5paper,hmargin={3cm,0.8in},
%]{geometry}

% Personal definitions
\newcommand{\dvec}[1]{\dot{\vec{#1}}}
\newcommand{\grad}{\vec{\nabla}}
\newcommand{\intV}[1]{\int_{-\infty}^{\infty} #1 d^3x}
\newcommand{\intVdot}[1]{\int_{-\infty}^{\infty} #1 d^3\dot{x}}
\newcommand{\intVVdot}[1]{\int_{-\infty}^{\infty}\int_{-\infty}^{\infty} #1 d^3xd^3\dot{x}}


\begin{document}

\title{A Derivation of the Conservation Equations for a Viscus Fluid}% Force line breaks with \

\author{Luke A. Siemens}
\email{lsiemens@uvic.ca}

\date{\today}

\maketitle

\section{State Space Dynamics}

Assuming the dynamics a particle is defined by the acceleration and that the acceleration of the particles is given by the potential $\phi$ such that $\vec{a} = \vec{\nabla}\cdot\phi$, then the equations of motion become
\[
\frac{d}{dt}\begin{pmatrix} \vec{x} \\ \dvec{x} \end{pmatrix}=\begin{pmatrix} \dvec{x} \\ -\grad\phi \end{pmatrix}
\]

For a system of $n$ particles where the $i^{\text{th}}$ particle is located at $\vec{x}_i$, the potential can in principle depend on the location of all of the particles, such that the potential for the $i^{\text{th}}$ particle is $\phi_i = \phi({\vec{x}_i, \vec{x}_1, \vec{x}_2, ... \vec{x}_i ... , \vec{x}_{n-1}, \vec{x}_n})$. Then for this system the equations of motion for each particle is given by the system of equations
\begin{equation}
\frac{d}{dt}\begin{pmatrix} \vec{x}_i \\ \dvec{x}_i \end{pmatrix}=\begin{pmatrix} \dvec{x}_i \\ -\partial_{\vec{x}_i}\phi_i \end{pmatrix}
\label{discrete_system_dynamics}
\end{equation}

Moving over to a state space description of the system, the state space has six coordinates given by the orthogonal coordinate vectors $\vec{x}$ and $\dvec{x}$. Given a time dependent density distribution defined over state space $\sigma=\sigma(\vec{x}, \dvec{x}, t)$, such that the total mass at the time $t$ is $M(t)=\intVVdot{\sigma(\vec{x}, \dvec{x}, t)}$. The system of discrete particles described by equation \eqref{discrete_system_dynamics} then be described by the distribution
\[
\sigma(\vec{x}, \dvec{x}, t) = m\sum^n_{i=0}\delta(\vec{x} - \vec{x}_i)\cdot\delta(\dvec{x} - \dvec{x}_i)
\]
where $m$ is the mass of the particles, and $\delta(\vec{x})$ is the Dirac delta distribution in 3-space. For this system the six component state space velocity is $\vec{v}=\left\langle\dvec{x}, -\grad\phi(\vec{x}, \rho, t)\right\rangle$, where the potential $\phi$ is defined such that at $\vec{x}_i$ the potential is $\phi(\vec{x}_i, \rho(\vec{x}, t), t)=\phi_i$ and $\rho(\vec{x}, t)=\intVdot{\sigma(\vec{x}, \dvec{x}, t)}$ . Assuming the number of particles in the distribution is constant, then $\frac{dM}{dt}=0$. Since there are no sources or sinks for the particles, the density distribution is constrained by the continuity equation
\[
\frac{d}{dt}\int_{V}\sigma(\vec{x}, \dvec{x}, t)d^3xd^3\dot{x}+\oint_{\partial V}\sigma(\vec{x}, \dvec{x}, t)\vec{v}\cdot d\vec{a}=0
\]
where $V$ is an arbitrary volume in state space, $\partial V$ is the surface of the arbitrary volume, $\vec{v}$ is the state space velocity and $d\vec{a}$ is a surface element in state space. Rearranging the terms and using the divergence theorem the continuity equation can be rewritten in the form
\[
\int_V\left(\partial_t\sigma + \grad\cdot\left(\sigma\vec{v}\right)\right)d^3xd^3\dot{x}=0
\]

Since the integral equals zero over any arbitrary volume $V$, then the integrand must be zero
\[
\partial_t \sigma + \grad\cdot\left(\sigma\vec{v}\right)=0
\]

Finally the independence of $\vec{x}$ and $\dvec{x}$ can be used to rewrite the continuity equation in the final form, in this case it is written using Einstein notation
\begin{equation}
\partial_t \sigma + \dot{x}_i\partial_{x_i}\sigma-\left(\partial_{x_i}\phi\right)\partial_{\dot{x}_i}\sigma=0
\label{state_space_continuity}
\end{equation}

The potential $\phi(\vec{x}, \rho, t)$ can be split into two components and external potential field $\phi_{\text{Ext}}(\vec{x}, t)$ and an internal field $\phi_{\text{Int}}(\vec{x}, \rho, t)$ where the internal component of the field is responsible for particle interactions. So then equation \eqref{state_space_continuity} is the Boltzmann Transport Equation where $\left(\partial_t \sigma\right)_{\text{Collision}} = \left(\partial_{x_i}\phi_{\text{Int}}\right)\partial_{\dot{x}_i}\sigma$ and $\phi$ is $\phi = \phi_{\text{Ext}}$,

\[
\partial_t \sigma + \dot{x}_i\partial_{x_i}\sigma-\left(\partial_{x_i}\phi\right)\partial_{\dot{x}_i}\sigma=\left(\partial_t \sigma\right)_{\text{Collision}}
\]

While the derivation of equation \eqref{state_space_continuity} was motivated using a discrete collection of particles the equation is not restricted to systems of discrete particles. As long as the dynamics in state space is determined by $\vec{v}=\left\langle\dvec{x}, -\grad\phi(\vec{x}, \rho, t)\right\rangle$ and there are no sources or sinks for the state space density, then any state space density function or distribution is described by equation \eqref{state_space_continuity}.

Assuming the motion of individual particles, in a fluid described by equation \eqref{discrete_system_dynamics} and assuming that the number and mass of the particles is invariant, then the dynamics of the state space distribution is described by equation \eqref{state_space_continuity}. If these assumptions are true for any fluid, then equation \eqref{state_space_continuity} must be capable of reproducing the behavior of fluid mechanics.



\section{Fluid Mechanics Mass Equation: Conservation of Mass}
The state space density function $\sigma$ is defined such that the total mass $M$ is $M=\intVVdot{\sigma(\vec{x}, \dvec{x}, t)}$. Define the mass density $\rho$ as $\rho(\vec{x}, t)=\intVdot{\sigma(\vec{x}, \dvec{x}, t)}$. Also define the bulk momentum $\rho\vec{u}$ as $\rho(\vec{x}, t) u_i(\vec{x}, t)=\intVdot{\dot{x}_i\sigma(\vec{x}, \dvec{x}, t)}=\intVdot{\dot{x}_i\sigma}$. The integral of equation \eqref{state_space_continuity} over velocity space is
\[
\intVdot{\left(\partial_t \sigma + \dot{x}_i\partial_{x_i}\sigma-\left(\partial_{x_i}\phi\right)\partial_{\dot{x}_i}\sigma\right)}=0
\]

Using the definition of $\rho$ and $\rho u_i$ this equation can be written in the form,
\[
\partial_t\rho + \partial_{x_i}\left(u_i\rho\right)-\left(\partial_{x_i}\phi\right)\oint{\sigma da_i}=0
\]

where $da_i$ is the $i^{\text{th}}$ component of the surface element, and $\oint{\sigma da_i}$ is the surface integral over all of velocity space. Assuming $\sigma(\vec{x}, \dvec{x}, t)$ drops to zero faster than $\lvert\dvec{x}\rvert^2$ then the surface integral converges to zero. Given the surface integral does converge to zero, the resulting equation is, 

\begin{equation}
\partial_t\rho + \partial_{x_i}\left(u_i\rho\right)=0
\label{conservation_of_mass}
\end{equation}

which is the conservation of mass equation from fluid mechanics.

\section{Conservation of Momentum}
To get an equation for the conservation of momentum, multiply equation \eqref{state_space_continuity} by $\dvec{x}$ before integrating over velocity space.
\[
\intVdot{\dot{x}_i\left(\partial_t \sigma + \dot{x}_j\partial_{x_j}\sigma-\left(\partial_{x_j}\phi\right)\partial_{\dot{x}_j}\sigma\right)}=0
\]

Using the definition of $\rho$ and $\rho u_i$ this equation can be written in the form,
\[
\begin{split}
& \partial_t\left(u_i\rho\right) + \partial_{x_j}\left(\intVdot{\dot{x}_i\dot{x}_j\sigma}\right) \\ & - \left(\partial_{x_j}\phi\right)\oint\dot{x}_i\sigma da_j + \left(\partial_{x_j}\phi\right)\rho\delta_{i j}=0
\end{split}
\]

where $\oint\dot{x}_i\sigma da_j$ is a surface integral over all of velocity space and $\delta_{ij}$ is the Kronecker delta function. Assuming $\sigma(\vec{x}, \dvec{x}, t)$ drops to zero faster than $\lvert\dvec{x}\rvert^3$ then the surface integral converges to zero. Given the surface integral does converge to zero then the result is,

\begin{equation}
\partial_t\left(u_i\rho\right) + \partial_{x_j}\left(\intVdot{\dot{x}_i\dot{x}_j\sigma}\right) + \rho\partial_{x_i}\phi=0
\label{conservation_of_momentum}
\end{equation}

If a term of $\rho u_i u_j$ is pulled out from $\intVdot{\dot{x}_i\dot{x}_j\sigma}$ and simplified then equation \eqref{conservation_of_momentum} can be put in a form analogous to the Navier-Stokes equations.

\section{Conservation of Energy}
The total energy of the system is $E_{\text{tot}}=\intVVdot{\left(\frac{1}{2}\dot{x}_i^2 + \phi(\vec{x}, \rho, t)\right)\sigma(\vec{x}, \dvec{x}, t)}$. In order to derive an equation for the conservation of energy equation in position space, equation \eqref{state_space_continuity} must be multiplied by $\frac{1}{2}\dot{x}_i^2 + \phi$ before integrating. 

\subsection{The Kinetic Energy Component}
Multiplying equation \eqref{state_space_continuity} by $\frac{1}{2}\dot{x}_i^2$ then integrating over velocity space, the resulting relation is
\[
\intVdot{\frac{1}{2}\dot{x}_i^2\left(\partial_t \sigma + \dot{x}_j\partial_{x_j}\sigma-\left(\partial_{x_j}\phi\right)\partial_{\dot{x}_j}\sigma\right)}=0
\]

Using the definition of $\rho$ and $\rho u_i$ this equation can be written in the form,
\[
\begin{split}
& \partial_t\left(\intVdot{\frac{1}{2}\dot{x}_i^2\sigma}\right) + \partial_{x_j}\left(\intVdot{\frac{1}{2}\dot{x}_i^2\dot{x}_j\sigma}\right) \\ & -  \left(\partial_{x_j}\phi\right)\left(\oint\left(\frac{1}{2}\dot{x}_i^2\sigma\right)da_j - \intVdot{\sigma\dot{x}_j}\right)=0
\end{split}
\]
where $\oint\left(\frac{1}{2}\dot{x}_i^2\sigma\right)da_j$ is a surface integral over all of velocity space. Assuming $\sigma(\vec{x}, \dvec{x}, t)$ drops to zero faster than $\lvert\dvec{x}\rvert^4$ then the surface integral converges to zero. Evaluating integrals and rearranging the equation given the surface integral does converge to zero, results in the equation,

\begin{equation}
\partial_t\left(\intVdot{\frac{1}{2}\dot{x}_i^2\sigma}\right) + \partial_{x_j}\left(\intVdot{\frac{1}{2}\dot{x}_i^2\dot{x}_j\sigma}\right) + u_j\rho\partial_{x_j}\phi=0
\label{incomplete_conservation_of_energy_kinetic}
\end{equation}

\subsection{Total Energy Density Dynamics}
The energy density $E$ is $E=\intVdot{\left(\frac{1}{2}\dot{x}_i^2 + \phi(\vec{x}, \rho, t)\right)\sigma(\vec{x}, \dvec{x}, t)}$. Note that multiplying equation \eqref{state_space_continuity} by $\phi$ and then integrating over velocity space results in equation \eqref{conservation_of_mass} multiplied by $\phi$. Adding the two energy components produces the equation,
\[
\begin{split}
& \partial_t\left(\intVdot{\left(\frac{1}{2}\dot{x}_i^2 + \phi\right)\sigma}\right) - \rho\partial_t\phi \\& + \partial_{x_j}\left(\intVdot{\frac{1}{2}\dot{x}_i^2\dot{x}_j\sigma} + u_j\rho\phi\right)=0
\end{split}
\]

Using the definition of energy density we can rewrite this equation as, 
\begin{equation}
\partial_t E + \partial_{x_j}\left(\intVdot{\frac{1}{2}\dot{x}_i^2\dot{x}_j\sigma} + u_j\rho\phi\right) - \rho\partial_t\phi=0
\label{conservation_of_energy}
\end{equation}

Equation \eqref{conservation_of_energy} describes the dynamics of the total energy density. This equation can be written in a more standard form if we pull out the terms $u_i\intVdot{\dot{x}_i\dot{x}_j\sigma} - \frac{1}{2}u_i^2u_j\rho$ from the integral $\intVdot{\frac{1}{2}\dot{x}_i^2\dot{x}_j\sigma}$ and then simplify.

\section{Fluid Model: Gaussian Distribution}
Lets assume that the velocity distribution $\sigma'/\rho'$ is simply a normalized Gaussian distribution. Then the state space distribution $\sigma'$ can be written as,

\begin{equation}
\sigma'(\vec{x}, \dvec{x}) = \rho'\left(\frac{m}{2\pi kT}\right)^{3/2}e^{-m\left(x_i - u'_i\right)^2/2kT}
\label{gaussian_state_space}
\end{equation}

Where $m$ is the mass of the particles the fluid is made from, $k$ is Boltzmann's constant and $T=T(\vec{x}, t)$ is the temperature. Also assume that the potential $\phi$ only explicitly depends on $\vec{x}$ and $\rho'$. So in this model $\phi=\phi(\vec{x}, \rho')$. All of the integrals in this derivation of fluid mechanics can be expressed as moments of the state space distribution. Before we calculate the moments of this distribution let us define some notation to simplify the expressions and integrals.

\subsection{Notation}
Let $\left(a^n\right)_{i_1i_2\cdots i_n}$ represent the repeated tensor product of $n$ copies of the tensor $a_i$ such that $\left(a^n\right)_{i_1i_2\cdots i_n}=a_{i_1}a_{i_2}\cdots a_{i_{n-1}}a_{i_n}$. Also let $\left[a^n\right]_{i_1i_2\cdots i_n}$ represent the sum of every permutation of indices applied to the tensor $\left(a^n\right)_{i_1i_2\cdots i_n}$ such that $\left[a^n\right]_{i_1i_2\cdots i_n}=a_{i_1}\left[a^{n-1}\right]_{i_2i_3\cdots i_n} + a_{i_2}\left[a^{n-1}\right]_{i_1i_3\cdots i_n} + \cdots + a_{i_n}\left[a^{n-1}\right]_{i_1i_2\cdots i_{n-1}}$. Given this notation if $u$ is a one index tensor and $\delta$ is the Kronecker delta symbol then using this notation $\left[u\delta\right]_{ijk}=u_i\left[\delta\right]_{jk} + u_j\left[\delta\right]_{ik} + u_k\left[\delta\right]_{ij}=2\left(u_i\delta_{jk} + u_j\delta_{ik} + u_k\delta_{ij}\right)$.

\subsection{Moments of a Gaussian}
Using this notation we can express the tensor describing the $n$th moment of a Gaussian state space distribution $\sigma=\rho\left(a \pi\right)^{-3/2}e^{-\left(\dot{x}_i-u_i\right)^2/a}$ as $A_{i_1i_2\cdots i_n}=\intVdot{\left(\dot{x}^n\right)_{i_1i_2\cdots i_n}\sigma}$. This integral can be solved by using the substitution $w_i=x_i - u_i$ and solving for one component of $A$ by pairing sequential indices and using that to reconstruct the full tensor. Solving for the moments of the Gaussian state space distribution results in the equation,

\begin{widetext}
\begin{equation}
A_{i_1i_2\cdots i_n}=\sum_{p=0}^n\rho\frac{a^{\left(n-p\right)/2}}{2^{n-p}p!\left(\left(n-p\right)/2\right)!}\left[u^p\delta^{\left(n-p\right)/2}\right]_{i_1i_2\cdots i_n}
\begin{cases}
0, & \text{if $n-p$ is odd}\\
1 &  \text{else}
\end{cases}
\label{gaussian_moments}
\end{equation}
\end{widetext}

\subsection{Dynamical Equation for a Gaussian}
Using equation \eqref{gaussian_moments} to evaluate the moments of the state space distribution \eqref{gaussian_state_space} produces the following equations,
\[
\intVdot{\sigma'}=\rho'
\]

\[
\intVdot{\dot{x}_i\sigma'}=\rho' u'_i
\]

\[
\intVdot{\left(\dot{x}^2\right)_{ij}\sigma'}=\frac{\rho'kT}{m}\delta_{ij} + \rho'\left(u'^2\right)_{ij}
\]

\[
\intVdot{\left(\dot{x}^3\right)_{ijk}\sigma'}=\frac{\rho'kT}{2m}\left[u'\delta\right]_{ijk} + \rho'\left(u'^3\right)_{ijk}
\]

Then the energy density $E'=\intVdot{\left(\frac{1}{2}\dot{x}_i^2 + \phi\right)\sigma'}=\rho'\frac{3kT}{2m}+\rho'\frac{1}{2}{u'}_i^2+\rho'\phi$. Now that all of the necessary integrals have been evaluated the conservation equations for the state space distribution \eqref{gaussian_state_space} are then,

\[
\partial_t \rho' + \partial_{x_i}\left(\rho'u'_i\right)=0
\]

\[
\partial_t \left(\rho' u'_i\right) + \partial_{x_i}\left(\rho'\frac{kT}{m}\right) + \partial_{x_j}\left(\rho'u'_iu'_j\right) + \rho'\partial_{x_i}\phi = 0
\]

\[
\partial_t E' + \partial_{x_i}\left(E'u'_i + \rho'\frac{kT}{m}u'_i\right)-\rho'\partial_t\phi = 0
\]

So the conservation equation resulting from using a Gaussian distribution as the state space distribution produces the equations of fluid dynamics describing a compressible inviscid fluid without thermal conduction.

\section{Fluid model: Perturbed Gaussian Distribution}
When constraining the state space distribution to a Gaussian distribution it captures some aspects of fluid mechanics but lacks higher order effects because of the symmetry of the distribution. This symmetry can be broken by perturbing the initial distribution. The fundamental dynamics of the system is described by equation \eqref{state_space_continuity}, so let us use that equation to perturb the initial state space distribution \eqref{gaussian_state_space} by evolving the equation forward in time for some short time $\Delta t$. Let the new state space distribution $\sigma$ be $\sigma=\sigma' + \Delta t\partial_t \sigma'$, where $\partial_t \sigma'=-\dot{x}_i\partial_{x_i}\sigma'+\left(\partial_{x_i}\phi\right)\partial_{\dot{x}_i}\sigma'$. Now to simplify this problem let us drop the terms $\partial_{x_i}\rho'$, $\partial_{x_i}T$ and $\partial_{x_i}\phi$ when computing $\partial_t \sigma'$. Then the new state space distribution can be written as,

\begin{equation}
\sigma = \sigma'\left(1-\Delta t\frac{m}{kT}\left(\partial_{x_i}u'_j\right)\dot{x}_i\left(\dot{x}_j-u'_j\right)\right)
\label{perturbed_gaussian_distribution}
\end{equation}

Note that all of the moments of this new distribution can be found using equation \eqref{gaussian_moments}. The moments of this distribution are,

\begin{widetext}

\[
\intVdot{\sigma'}=\rho'\left(1-\Delta t\partial_{x_i}u'_i\right)
\]

\[
\intVdot{\dot{x}_i\sigma'}=\rho' u'_i-\rho'\Delta t\partial_{x_j}\left(u'_iu'_j\right)
\]

\[
\intVdot{\left(\dot{x}^2\right)_{ij}\sigma'}=\frac{\rho'kT}{m}\delta_{ij} + \rho'\left({u'}^2\right)_{ij}-\rho'\frac{kT}{m}\Delta t\left(\partial_{x_k}u'_k\delta_{ij} + \partial_{x_i}u'_j + \partial_{x_j}u'_i\right)-\rho'\Delta t\partial_{x_k}\left({u'}^3\right)_{ijk}
\]

\[
\begin{split}
\intVdot{\left(\dot{x}^3\right)_{ijk}\sigma'} & = \frac{\rho'kT}{2m}\left[u'\delta\right]_{ijk} + \rho'\left(u'^3\right)_{ijk} - \rho'\Delta t\partial_{x_l}\left({u'}^4\right)_{ijkl} \\ & -\rho'\frac{kT}{m}\Delta t\left(\delta_{ij}\partial_{x_l}\left(u'_ku'_l\right)+\delta_{ik}\partial_{x_l}\left(u'_ju'_l\right) + \delta_{jk}\partial_{x_l}\left(u'_iu'_l\right) + \partial_{x_i}\left(u'_ju'_k\right) + \partial_{x_j}\left(u'_iu'_k\right) + \partial_{x_k}\left(u'_iu'_j\right)\right)
\end{split}
\]

Using the moments of the perturbed distribution the conservation equations can be written as,

\[
\partial_t \rho + \partial_{x_i}\left(\rho u_i\right)=0
\]

\[
\partial_t \left(\rho u_i\right) + \partial_{x_j}A_{ij} + \rho\partial_{x_i}\phi = 0
\]

\[
\partial_t E + \partial_{x_i}\left(B_i + \rho\phi u_i\right)-\rho\partial_t\phi = 0
\]

Where the density $\rho$ is $\rho=\rho'\left(1-\Delta t\partial_{x_i}u'_i\right)$, the bulk momentum $\rho u_i$ is $\rho u_i=\rho' u'_i-\rho'\Delta t\partial_{x_j}\left(u'_iu'_j\right)$, the energy $E$ is $E=E'-\rho'\frac{5kT}{2m}\Delta t\partial_{x_i}u'_i-\frac{1}{2}\rho'\Delta t\partial_{x_j}\left({u'}_i^2u'_j\right)-\rho'\phi\Delta t\partial_{x_i}u'_i$, the tensor $A_{ij}$ is $A_{ij}=\frac{\rho'kT}{m}\delta_{ij} + \rho'\left({u'}^2\right)_{ij}-\rho'\frac{kT}{m}\Delta t\left(\partial_{x_k}u'_k\delta_{ij} + \partial_{x_i}u'_j + \partial_{x_j}u'_i\right)-\rho'\Delta t\partial_{x_k}\left({u'}^3\right)_{ijk}$ and the vector $B_i$ is $B_i=\frac{5\rho'kT}{2m}u'_i + \frac{1}{2}\rho'{u'}_j^2u'_i-\rho'\frac{kT}{2m}\Delta t\left(7\partial_{x_j}\left(u'_iu'_j\right) + \partial_{x_i}{u'}_j^2\right)-\frac{1}{2}\rho'\Delta t\partial_{x_k}\left({u'}_j^2u'_iu'_k\right)$.

Let us assume that and that $u'_i$ is small, the relative fluctuations in $\rho'$, $u'_i$ and $T$ are small and that $\partial_{x_i}u'_i=0$. Given these assumptions after dropping the higher order terms the conservation equations can be written as

\[
\partial_t\left(\rho'\right) + \left(u'_i - u'_j\Delta t\partial_{x_j}u'_i\right)\partial_{x_i}\rho' - \rho'\Delta t\partial_{x_i}u'_j\partial_{x_j}u'_i=0
\]

\[
\rho'\left(\partial_t\left(u'_i - u'_j\Delta t\partial_{x_j}u'_i\right) + u'_j\partial_{x_j}u'_i\right) + \partial_{x_j}\left(\frac{\rho'kT}{m}\right) - \partial_{x_j}\left(\rho'\frac{kT}{m}\Delta t\left(\partial_{x_i}u'_j + \partial_{x_j}u'_i\right)\right) + \rho'\partial_{x_i}\phi = 0
\]

\[
\partial_t E' + \partial_{x_i}\left(\left(E' + \frac{\rho'kT}{m}\right)\left(u'_i - u'_j\Delta t\partial_{x_j}u'_i\right) - \rho'\frac{kT}{m}\Delta t\left(\partial_{x_j}u'_i + \partial_{x_i}u'_j\right)u'_j\right)-\rho'\partial_t\phi = 0
\]

This equation can be simplified again if we assume the term $\Delta t\partial_{x_i}u'_j\partial_{x_j}u'_i$ is negligible. Now the conservation equations can be written as,

\[
\partial_t\left(\rho'\right) + \left(u'_i - u'_j\Delta t\partial_{x_j}u'_i\right)\partial_{x_i}\rho'=0
\]

\[
\rho'\left(\partial_t\left(u'_i - u'_j\Delta t\partial_{x_j}u'_i\right) + u'_j\partial_{x_j}u'_i\right) + \partial_{x_j}\left(\frac{\rho'kT}{m}\right) - \partial_{x_j}\left(\rho'\frac{kT}{m}\Delta t\left(\partial_{x_j}u'_i + \partial_{x_i}u'_j\right)\right) + \rho'\partial_{x_i}\phi = 0
\]

\[
\partial_t E' + \left(u'_i - u'_j\Delta t\partial_{x_j}u'_i\right)\partial_{x_i}\left(E' + \frac{\rho'kT}{m}\right) - \partial_{x_i}\left(\rho'\frac{kT}{m}\Delta t\left(\partial_{x_j}u'_i + \partial_{x_i}u'_j\right)u'_j\right) - \rho'\partial_t\phi = 0
\]

So perturbing to the state space distribution \eqref{gaussian_state_space} can produce conservation equations which include higher order terms. For the particular perturbation \eqref{perturbed_gaussian_distribution} the conservation equations describe a viscous fluid, and given some assumptions the equations can be put into a form comparable to the standard equations for an incompressible viscous fluid.
\end{widetext}

\section{Complete First Order Perturbation}
I have demonstrated that perturbing a Gaussian distribution with only the $\partial_{x_i}u'_j$ component of $\partial_t \sigma'=-\dot{x}_i\partial_{x_i}\sigma'+\left(\partial_{x_i}\phi\right)\partial_{\dot{x}_i}\sigma'$ leads to viscus terms in the resulting conservation equations. Based on my preliminary investigation into how the remaining components ($\partial_{x_i}\rho$, $\partial_{x_i}T$ and $\partial_{x_i}\phi$) effect the perturbation of the Gaussian state space distribution \eqref{gaussian_state_space} I expect that a full first order perturbation of the distribution will produce an additional three sets of terms in the conservation equations. As part of these new terms I expect to find terms including density diffusion terms from $\partial_{x_i}\rho$, thermal conduction terms from $\partial_{x_i}T$ and terms resulting from $\partial_{x_i}\phi$ due to particle interactions mediated by the potential $\phi\left(\vec{x}, t, \rho\right)$. For example if $\phi\left(\vec{x}, t, \rho\right)=\int_{-\infty}^{\infty}f(|\vec{x}-\vec{x'}|)\rho(\vec{x'})d^3x'$ where $f(r)$ models the interaction between particles, then depending on the structure of local minima and maxima in the function $f(r)$ I expect that the net effect of the terms resulting from $\partial_{x_i}\phi$ will be to introduce phase transitions into the conservation equations.

Also note that if the initial state space distribution is of the form of equation \eqref{gaussian_state_space} then the all of the moments of the perturbed distribution can be solved using equation \eqref{gaussian_moments} if the perturbed distribution is constructed from $n$th perturbation the initial distribution using equation \eqref{state_space_continuity}.
\end{document}
%
% ****** End of file apssamp.tex ******
