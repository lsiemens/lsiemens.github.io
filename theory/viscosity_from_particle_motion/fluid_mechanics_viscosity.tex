% ****** Start of file apssamp.tex ******
%
%   This file is part of the APS files in the REVTeX 4.1 distribution.
%   Version 4.1r of REVTeX, August 2010
%
%   Copyright (c) 2009, 2010 The American Physical Society.
%
%   See the REVTeX 4 README file for restrictions and more information.
%
% TeX'ing this file requires that you have AMS-LaTeX 2.0 installed
% as well as the rest of the prerequisites for REVTeX 4.1
%
% See the REVTeX 4 README file
% It also requires running BibTeX. The commands are as follows:
%
%  1)  latex apssamp.tex
%  2)  bibtex apssamp
%  3)  latex apssamp.tex
%  4)  latex apssamp.tex
%
\documentclass[%
%reprint,
%superscriptaddress,
%groupedaddress,
%unsortedaddress,
%runinaddress,
%frontmatterverbose, 
preprint,
%showpacs,preprintnumbers,
%nofootinbib,
%nobibnotes,
%bibnotes,
 amsmath,amssymb,
 aps,
%pra,
%prb,
%rmp,
%prstab,
%prstper,
%floatfix,
]{revtex4-1}

\usepackage{graphicx}% Include figure files
\usepackage{dcolumn}% Align table columns on decimal point
\usepackage{bm}% bold math
\usepackage{amsmath}% better dot placment
\usepackage{systeme}% systemes of equations
%\usepackage{hyperref}% add hypertext capabilities
%\usepackage[mathlines]{lineno}% Enable numbering of text and display math
%\linenumbers\relax % Commence numbering lines

%\usepackage[showframe,%Uncomment any one of the following lines to test 
%%scale=0.7, marginratio={1:1, 2:3}, ignoreall,% default settings
%%text={7in,10in},centering,
%%margin=1.5in,
%%total={6.5in,8.75in}, top=1.2in, left=0.9in, includefoot,
%%height=10in,a5paper,hmargin={3cm,0.8in},
%]{geometry}

% Personal definitions
\newcommand{\dvec}[1]{\dot{\vec{#1}}}
\newcommand{\grad}{\vec{\nabla}}
\newcommand{\intV}[1]{\int_{-\infty}^{\infty} #1 d^3x}
\newcommand{\intVdot}[1]{\int_{-\infty}^{\infty} #1 d^3\dot{x}}
\newcommand{\intVVdot}[1]{\int_{-\infty}^{\infty}\int_{-\infty}^{\infty} #1 d^3xd^3\dot{x}}


\begin{document}

\title{Derivation of viscosity}% Force line breaks with \

\author{L. Siemens}
\email{lsiemens@uvic.ca}

\date{\today}

\begin{abstract}
I derive a single PDE that describes the dynamics of fluids in state space. Then the similarity between that PDE and equations of fluid mechanics is demonstrated by using it to deriving a set of three equations analogous to the mass, internal energy and Navier-Stokes equations.  Finally I demonstrate that for a fluid with particles following the Maxwell-Boltzmann distribution the set of analogous equations reduces to the equations of inviscid flow.
\end{abstract}

\maketitle

\section{Introduction}

Fluids consist of a large number of interacting particles, presumably the fluid mechanics observed at a macroscopic level is a result of the interactions occurring at a microscopic level. If the dynamics of each particle was know it should be possible in principle to determine the macroscopic dynamics of the fluid from the collective motion of the particles. 

\section{State space dynamics}

Assuming the dynamics a particle is defined by the acceleration and that the acceleration of the particles is given by the potential $\phi$ such that $\vec{a} = \vec{\nabla}\cdot\phi$, then the equations of motion become
\[
\frac{d}{dt}\begin{pmatrix} \vec{x} \\ \dvec{x} \end{pmatrix}=\begin{pmatrix} \dvec{x} \\ -\grad\phi \end{pmatrix}
\]

For a system of $n$ particles where the $i^{\text{th}}$ particle is located at $\vec{x}_i$, the potential can in principle depend on the location of all of the particles, such that the potential for the $i^{\text{th}}$ particle is $\phi_i = \phi({\vec{x}_i, \vec{x}_1, \vec{x}_2, ... \vec{x}_i ... , \vec{x}_{n-1}, \vec{x}_n})$. Then for this system the equations of motion for each particle is given by the system of equations
\begin{equation}
\frac{d}{dt}\begin{pmatrix} \vec{x}_i \\ \dvec{x}_i \end{pmatrix}=\begin{pmatrix} \dvec{x}_i \\ -\partial_{\vec{x}_i}\phi_i \end{pmatrix}
\label{discrete_system_dynamics}
\end{equation}

Moving over to a state space description of the system, the state space has six coordinates given by the orthogonal coordinate vectors $\vec{x}$ and $\dvec{x}$. Given a time dependent density distribution defined over state space $\sigma=\sigma(\vec{x}, \dvec{x}, t)$, such that the total mass at the time $t$ is $M(t)=\intVVdot{\sigma(\vec{x}, \dvec{x}, t)}$. The system of discrete particles described by equation \eqref{discrete_system_dynamics} then be described by the distribution
\[
\sigma(\vec{x}, \dvec{x}, t) = m\sum^n_{i=0}\delta(\vec{x} - \vec{x}_i)\cdot\delta(\dvec{x} - \dvec{x}_i)
\]
where $m$ is the mass of the particles, and $\delta(\vec{x})$ is the Dirac delta distribution in 3-space. For this system the six component state space velocity is $\vec{v}=\left\langle\dvec{x}, -\grad\phi(\vec{x}, \rho, t)\right\rangle$, where the potential $\phi$ is defined as $\phi(\vec{x}_i, \rho, t)=\phi_i$ and $\rho(\vec{x}, t)=\intVdot{\sigma(\vec{x}, \dvec{x}, t)}$ . Assuming the number of particles in the distribution is constant, then $\frac{dM}{dt}=0$. Since there are no sources or sinks for the particles, the density distribution is constrained by the continuity equation
\[
\frac{d}{dt}\int_{V}\sigma(\vec{x}, \dvec{x}, t)d^3xd^3\dot{x}+\oint_{\partial V}\sigma(\vec{x}, \dvec{x}, t)\vec{v}\cdot d\vec{a}=0
\]
where $V$ is an arbitrary volume in state space, $\partial V$ is the surface of the arbitrary volume, $\vec{v}$ is the state space velocity and $d\vec{a}$ is a surface element in state space. Rearranging the terms and using the divergence theorem the continuity equation can be rewritten in the form
\[
\int_V\left(\partial_t\sigma + \grad\cdot\left(\sigma\vec{v}\right)\right)d^3xd^3\dot{x}=0
\]

Since the integral equals zero over any arbitrary volume $V$, then the integrand must be zero
\[
\partial_t \sigma + \grad\cdot\left(\sigma\vec{v}\right)=0
\]

Finally the independence of $\vec{x}$ and $\dvec{x}$ can be used to rewrite the continuity equation in the final form, in this case it is written using Einstein notation
\begin{equation}
\partial_t \sigma + \dot{x}_i\partial_{x_i}\sigma-\left(\partial_{x_i}\phi\right)\partial_{\dot{x}_i}\sigma=0
\label{state_space_continuity}
\end{equation}

While the derivation of equation \eqref{state_space_continuity} was motivated using a discrete collection of particles the equation is not restricted to systems of discrete particles. As long as the dynamics in state space is determined by $\vec{v}=\left\langle\dvec{x}, \grad\phi(\vec{x}, \rho, t)\right\rangle$ and there are no sources or sinks for the state space density, then any state space density function or distribution is described by equation \eqref{state_space_continuity}.

Assuming the motion of individual particles, in a fluid described by fluid mechanics, is described by equation \eqref{discrete_system_dynamics} and assuming that the number and mass of the particles is invariant, then the dynamics of the state space distribution is described by equation \eqref{state_space_continuity}. If these assumptions are true for any fluid, then equation \eqref{state_space_continuity} must be capable of reproducing the behavior of fluid mechanics.

\section{Fluid mechanics mass equation: Conservation of mass}
The state space density function $\sigma$ is defined such that the total mass $M$ is $M=\intVVdot{\sigma(\vec{x}, \dvec{x}, t)}$. Define the mass density $\rho$ as $\rho(\vec{x}, t)=\intVdot{\sigma(\vec{x}, \dvec{x}, t)}$. Also define the bulk momentum $\rho\vec{u}$ as $\rho(\vec{x}, t) u_i(\vec{x}, t)=\intVdot{\dot{x}_i\sigma(\vec{x}, \dvec{x}, t)}=\intVdot{\dot{x}_i\sigma}$. The integral of equation \eqref{state_space_continuity} over velocity space is
\[
\intVdot{\left(\partial_t \sigma + \dot{x}_i\partial_{x_i}\sigma-\left(\partial_{x_i}\phi\right)\partial_{\dot{x}_i}\sigma\right)}=0
\]

Splitting up the integral, pulling out factors and operations that are independent of $\dot{x}_i$ puts the equation into a form where some terms can be evaluated. Then applying the divergence theorem and evaluating the simplified integrals results in the equation
\[
\partial_t\rho + \partial_{x_i}\left(u_i\rho\right)-\left(\partial_{x_i}\phi\right)\oint{\sigma da_i}=0
\]
where $da_i$ is the $i^{\text{th}}$ component of the surface element, and $\oint{\sigma da_i}$ is the surface integral over all of velocity space. Assuming $\sigma(\vec{x}, \dvec{x}, t)$ drops to zero faster than $\lvert\dvec{x}\rvert^2$ then the surface integral converges to zero. Given the surface integral does converge to zero, the resulting equation is $\partial_t\rho + \partial_{x_i}\left(u_i\rho\right)=0$. When written in vector notation it becomes
\begin{equation}
\partial_t\rho + \grad\cdot\left(\vec{u}\rho\right)=0
\label{conservation_of_mass}
\end{equation}
which is the conservation of mass equation from fluid mechanics.

\section{Conservation of momentum}
To get an equation for the conservation of momentum, multiply equation \eqref{state_space_continuity} by $\dvec{x}$ before integrating over velocity space.
\[
\intVdot{\dot{x}_i\left(\partial_t \sigma + \dot{x}_j\partial_{x_j}\sigma-\left(\partial_{x_j}\phi\right)\partial_{\dot{x}_j}\sigma\right)}=0
\]

After simplifying the equation, applying the product rule and divergence theorem it can be written in the form
\[
\begin{split}
& \partial_t\left(u_i\rho\right) + \partial_{x_j}\left(\intVdot{\dot{x}_i\dot{x}_j\sigma}\right) \\ & - \left(\partial_{x_j}\phi\right)\oint\dot{x}_i\sigma da_j + \left(\partial_{x_j}\phi\right)\rho\delta_{i j}=0
\end{split}
\]
where $\oint\dot{x}_i\sigma da_j$ is a surface integral over all of velocity space and $\delta_{ij}$ is the Kronecker delta function. Assuming $\sigma(\vec{x}, \dvec{x}, t)$ drops to zero faster than $\lvert\dvec{x}\rvert^3$ then the surface integral converges to zero. Given the surface integral does converge to zero then the result is,

\begin{equation}
\partial_t\left(u_i\rho\right) + \partial_{x_j}\left(\intVdot{\dot{x}_i\dot{x}_j\sigma}\right) + \rho\partial_{x_i}\phi=0
\label{conservation_of_momentum}
\end{equation}

If we pulled out a term of $\rho u_i u_j$ from $\intVdot{\dot{x}_i\dot{x}_j\sigma}$ and simplified then equation \eqref{conservation_of_momentum} can be put in the form the Navier-Stokes equations.

\section{Conservation of energy}
The total energy of the system is $E_{\text{tot}}=\intVVdot{\left(\frac{1}{2}\dot{x}_i^2 + \phi(\vec{x}, \rho, t)\right)\sigma(\vec{x}, \dvec{x}, t)}$.  In order to derive an equation for the conservation of energy equation in position space, equation \eqref{state_space_continuity} must be multiplied by $\frac{1}{2}\dot{x}_i^2 + \phi$ before integrating. To make the derivation of the conservation of energy simpler we can use the linearity of integration to break the conservation of energy equation into a kinetic energy component and a potential energy component.

\subsection{The kinetic energy component}
Multiplying equation \eqref{state_space_continuity} by $\frac{1}{2}\dot{x}_i^2$ then integrating over velocity space, the resulting relation is
\[
\intVdot{\frac{1}{2}\dot{x}_i^2\left(\partial_t \sigma + \dot{x}_j\partial_{x_j}\sigma-\left(\partial_{x_j}\phi\right)\partial_{\dot{x}_j}\sigma\right)}=0
\]

After simplifying the equation, applying the product rule and divergence theorem, it can be written in the form
\[
\begin{split}
& \partial_t\left(\intVdot{\frac{1}{2}\dot{x}_i^2\sigma}\right) + \partial_{x_j}\left(\intVdot{\frac{1}{2}\dot{x}_i^2\dot{x}_j\sigma}\right) \\ & -  \left(\partial_{x_j}\phi\right)\left(\oint\left(\frac{1}{2}\dot{x}_i^2\sigma\right)da_j - \intVdot{\sigma\dot{x}_j}\right)=0
\end{split}
\]
where $\oint\left(\frac{1}{2}\dot{x}_i^2\sigma\right)da_j$ is a surface integral over all of velocity space. Assuming $\sigma(\vec{x}, \dvec{x}, t)$ drops to zero faster than $\lvert\dvec{x}\rvert^4$ then the surface integral converges to zero. Evaluating integrals and rearranging the equation given the surface integral does converge to zero, results in the equation,

\begin{equation}
\partial_t\left(\intVdot{\frac{1}{2}\dot{x}_i^2\sigma}\right) + \partial_{x_j}\left(\intVdot{\frac{1}{2}\dot{x}_i^2\dot{x}_j\sigma}\right) + u_j\rho\partial_{x_j}\phi=0
\label{incomplete_conservation_of_energy_kinetic}
\end{equation}

\subsection{The potential energy component}
Multiplying equation \eqref{state_space_continuity} by $\phi$ and then integrating over velocity space, the resulting relation is
\[
\intVdot{\phi\left(\partial_t \sigma + \dot{x}_j\partial_{x_j}\sigma-\left(\partial_{x_j}\phi\right)\partial_{\dot{x}_j}\sigma\right)}=0
\]

After rearranging and simplifying use the result that $\intVdot{\partial_{\dot{x}_j}\sigma}=\oint\sigma da_j = 0$, from the derivation for the conservation of mass equation, to get
\[
\phi\partial_t\rho + \phi\partial_{x_j}\left(u_j\rho\right)=0
\]

Applying the product rule produces the form
\begin{equation}
\partial_t\left(\phi\rho\right) - \rho\partial_t\phi + \partial_{x_j}\left(u_j\rho\phi\right) - u_j\rho\partial_{x_j}\phi=0
\label{incomplete_conservation_of_energy_potential}
\end{equation}


\subsection{Total energy density dynamics}
The total energy density $E$ is $E=\intVdot{\left(\frac{1}{2}\dot{x}_i^2 + \phi(\vec{x}, \rho, t)\right)\sigma(\vec{x}, \dvec{x}, t)}$. Adding the two energy components from equation \eqref{incomplete_conservation_of_energy_kinetic} and equation \eqref{incomplete_conservation_of_energy_potential} produces the equation
\[
\begin{split}
& \partial_t\left(\intVdot{\left(\frac{1}{2}\dot{x}_i^2 + \phi\right)\sigma}\right) - \rho\partial_t\phi \\& + \partial_{x_j}\left(\intVdot{\frac{1}{2}\dot{x}_i^2\dot{x}_j\sigma} + u_j\rho\phi\right)=0
\end{split}
\]

Using the definition of energy density we can rewrite this equation as, 
\begin{equation}
\partial_t E + \partial_{x_j}\left(\intVdot{\frac{1}{2}\dot{x}_i^2\dot{x}_j\sigma} + u_j\rho\phi\right) - \rho\partial_t\phi=0
\label{conservation_of_energy}
\end{equation}

Equation \eqref{conservation_of_energy} describes the dynamics of the total energy density. This equaiton can be written in a more standard form if we pull out the terms $u_i\intVdot{\dot{x}_i\dot{x}_j\sigma} - \frac{1}{2}u_i^2u_j\rho$ from the integral $\intVdot{\frac{1}{2}\dot{x}_i^2\dot{x}_j\sigma}$ and then simplify.

\section{Fluid model: Gaussian distribution}
Lets assume that the velocity distribution $\sigma/\rho$ is simply a normalized Gaussian distribution. Then the state space distribution $\sigma$ is $\sigma(\vec{x}, \dvec{x}) = \rho\left(\frac{m}{2\pi kT}\right)^{3/2}e^{-\frac{m\left(x_i - u_i\right)^2}{2kT}}$, where $m$ is the mass of the particles the fluid is made from, $k$ is Boltzmann's constant and $T=T(\vec{x}, t)$ is the thermodynamic temperature. Also assume that the potential $\phi$ only explicitly depends on $\vec{x}$ and $\rho$. So in this model $\phi=\phi(\vec{x}, \rho)$. All of the integrals in this derivation of fluid mechanics can be expressed as moments of the state space distribution. before we calculate the moments of the distribution $\sigma$ lets define sum notation to simplify the expressions.

\subsection{Notation}
Let $\left(a^n\right)_{i_1i_2\cdots i_n}$ represent the repeated tensor product of $n$ copies of the tensor $a_i$ such that $\left(a^n\right)_{i_1i_2\cdots i_n}=a_{i_1}a_{i_2}\cdots a_{i_{n-1}}a_{i_n}$. And let $\left[a^n\right]_{i_1i_2\cdots i_n}$ represent the sum of every permutation of indecies applied to the tensor $\left(a^n\right)_{i_1i_2\cdots i_n}$ such that $\left[a^n\right]_{i_1i_2\cdots i_n}=a_{i_1}\left[a^{n-1}\right]_{i_2i_3\cdots i_n} + a_{i_2}\left[a^{n-1}\right]_{i_1i_3\cdots i_n} + \cdots + a_{i_n}\left[a^{n-1}\right]_{i_1i_2\cdots i_{n-1}}$. For example if $u$ is a one index tensor $\left[u\delta\right]_{ijk}=2\left(u_i\delta_{jk} + u_j\delta_{ik} + u_k\delta_{ij}\right)$

\subsection{Moments of a Gaussian}
Using this notation we can express the tensor describing the nth moment $A_{i_1i_2\cdots i_n}$ of a Gaussian state space distribution $\sigma=\rho\left(a \pi\right)^{-3/2}e^{-\left(\dot{x}_i-u_i\right)^2/a}$ as $A_{i_1i_2\cdots i_n}=\intVdot{\left(\dot{x}^n\right)_{i_1i_2\cdots i_n}\sigma}$. This integral can be solved by using the substitution $w_i=x_i - u_i$ and solving for one term of $A$ by pairing indices and using that to reconstruct the full tensor. Solving for the moments of the distribution results in the equation,

\[
A_{i_1i_2\cdots i_n}=\intVdot{\left(\dot{x}^n\right)_{i_1i_2\cdots i_n}\sigma}=\sum_{p=0}^n\rho\frac{a^{\left(n-p\right)/2}}{2^{n-p}p!\left(\left(n-p\right)/2\right)!}\left[u^p\delta^{\left(n-p\right)/2}\right]_{i_1i_2\cdots i_n}
\begin{cases}
0, & \text{if $n$ is odd}\\
1 &  \text{else}
\end{cases}
\]

\subsection{Dynamical equation for a Gaussian}
Assuming the speed of particles in a fluid follow a Maxwell-Boltzmann distribution, then the velocity distribution of the fluid is a normalized Gaussian distribution in three dimensional velocity space. For a fluid element with a non-zero bulk velocity assume that the velocity distribution is simply a normalized Gaussian with a non-zero mean value. Given these assumptions the velocity distribution at any point is $\sigma(\vec{x}, \dvec{x}) / \rho(\vec{x}) = \left(\frac{m}{2\pi kT}\right)^{3/2}e^{-\frac{m\left(x_i - u_i\right)^2}{2kT}}$, where $m$ is the mass of the particles the fluid is made from, $k$ is Boltzmann's constant and $T=T(\vec{x}, t)$ is the thermodynamic temperature. Also assume that the potential $\phi$ only explicitly depends on $\vec{x}$ and $\rho$. So in this model $\phi=\phi(\vec{x}, \rho)$ and the state space density distribution is,
\begin{equation}
\sigma(\vec{x}, \dvec{x}) = \rho(\vec{x})\left(\frac{m}{2\pi kT}\right)^{3/2}e^{-\frac{m\left(x_i - u_i\right)^2}{2kT}}
\label{distribution_maxwell_boltzmann}
\end{equation}

As defined earlier the tensor $A$ is
\[
A_{ij} = -\frac{1}{2}\left(\intVdot{\left(\dot{x}_i - u_i\right)\dot{x}_j\sigma} + \intVdot{\left(\dot{x}_j - u_j\right)\dot{x}_i\sigma}\right)
\]

When $i$ is not equal to $j$ the integral is $A_{ij} = 0$, when $i$ is equal to $j$ the integral is $A_{ij} = -kT\rho/m$, so the full tensor is $A_{ij}=-kT\frac{\rho}{m}\delta_{ij}$. Using the ideal gas law $PV=NkT$, since $\rho/m=N/V$ where $N$ is the number of particles and $V$ is the volume, then $A_{ij} = -P\delta_{ij}$, $p=-\frac{1}{3}A_{ii}=P$ and $B_{ij}=A_{ij}-\frac{1}{3}A_{kk}\delta_{ij} = 0$.

The vector $\vec{C}$ is defined as $C_j = \intVdot{\frac{1}{2}v_i^2 v_j\sigma}$, evaluating the integral results in the equation $C_j = 0$. So given the state space density function \eqref{distribution_maxwell_boltzmann} the equations of fluid mechanics derived from equation \eqref{state_space_continuity} are
\[
\partial_t\rho + \grad\cdot\left(\vec{u}\rho\right)=0
\]

\[
\rho\left(\partial_t \vec{u} + \vec{u}\cdot\grad\vec{u}\right) = - \grad P - \rho\grad\phi
\]

\[
\partial_te + \grad\cdot\left(e \vec{u}\right) = \rho\vec{u}\cdot\grad\phi_{\text{in}} - P\grad\cdot\vec{u}
\]

These equations are equal to the conservation of mass, internal energy density and Navier-Stokes equations in the case of inviscid flow.

\end{document}
%
% ****** End of file apssamp.tex ******
